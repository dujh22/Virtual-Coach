\documentclass[sigconf]{acmart}

\AtBeginDocument{%
  \providecommand\BibTeX{{%
    Bib\TeX}}}

\setcopyright{acmlicensed}
\copyrightyear{2018}
\acmYear{2018}
\acmDOI{XXXXXXX.XXXXXXX}

\acmConference[Conference acronym 'XX]{Make sure to enter the correct
  conference title from your rights confirmation email}{June 03--05,
  2018}{Woodstock, NY}

\acmISBN{978-1-4503-XXXX-X/2018/06}

\begin{document}

\title{EDMIT: An End-to-End Agentic Framework for Enhanced Decision-Making in Interactive Motion Tutoring}

\author{Jinhua Du}

\email{dujh22@mails.tsinghua.edu.cn}
\orcid{0009-0008-4170-6452}
\affiliation{
  \institution{Tsinghua University}
  \city{Beijing}
  \country{China}
}

\author{Mufeng Xing}
\author{Ziheng Zhou}
\author{Ruilin Zhang}
\author{Zhongshi Liu}
\affiliation{
  \institution{Banlan Technology}
  \city{Suzhou}
  \country{China}
}

\author{Zexun Jiang}
\affiliation{
  \institution{School of Data Science and Intelligent Media Communication University of China,}
  \city{Beijing}
  \country{China}
}

\begin{abstract}
One-on-one tutoring has proven effective, yet current LLM-based educational systems still struggle to achieve individualized decision-making in multi-turn, human-AI co-performance settings. We focus on three open challenges: (i) how to construct usable data from scarce and unstructured coaching experience at low cost, (ii) how to design adaptive feedback mechanisms that accommodate individual differences during interaction, and (iii) whether AI-based coaching can match—or even surpass—human tutors. To address these, we propose EDMIT, an end-to-end agentic framework for interactive motion tutoring. EDMIT builds a seed-to-augmentation data pipeline from coaching trajectories, introduces a standardized decision ontology and executable decision chain, and leverages closed-loop feedback to dynamically adjust prompt granularity, exercise difficulty, and error-correction strategies during sessions. We implemented a prototype and conducted a double-blind controlled study with 10 users and 10 human coaches, comparing EDMIT against a prompt-based baseline under source-agnostic conditions. Results show that EDMIT achieves higher decision consistency with human coaches, significantly improves user satisfaction and acceptance, and reaches state-of-the-art performance relative to existing methods. These findings demonstrate that agent-driven human-AI co-performance can effectively bridge the gap of individualized decision-making in motion tutoring, offering empirical evidence and methodological insights for the design of future AI coaching systems.
\end{abstract}

\begin{CCSXML}
  <ccs2012>
  <concept>
  <concept_id>10003120.10003121.10003122.10003332</concept_id>
  <concept_desc>Human-centered computing~User models</concept_desc>
  <concept_significance>500</concept_significance>
  </concept>
  </ccs2012>
\end{CCSXML}

\ccsdesc[500]{Human-centered computing~User models}

\keywords{Interactive Motion Tutoring; Agentic Framework, Personalized Decision-Making, Large Language Models (LLMs), Data Augmentation, Feedback Loop, Decision Chain, Double-Blind Study, Empirical Evaluation}


\begin{teaserfigure}
  \includegraphics[width=\textwidth]{sampleteaser}
  \caption{Seattle Mariners at Spring Training, 2010.}
  \Description{Enjoying the baseball game from the third-base
  seats. Ichiro Suzuki preparing to bat.}
  \label{fig:teaser}
\end{teaserfigure}

\maketitle

\section{Introduction}

One-on-one human tutoring has long been recognized as a highly effective form of instruction \cite{chiLearningHumanTutoring2001}. With the rapid advancement of artificial intelligence—particularly the maturation of large language model (LLM)-based technologies—numerous intelligent systems have emerged to support end users in domains such as education \cite{wangLargeLanguageModels2024}, healthcare \cite{heSurveyLargeLanguage2025}, and sports science \cite{connorLargeLanguageModels2023}. More recently, the advent of deep reasoning models, exemplified by o1 \cite{openaiOpenAIO1System2024a} and DeepSeek \cite{deepseek-aiDeepSeekR1IncentivizingReasoning2025b}, has significantly strengthened the reasoning capabilities of intelligent systems. These models have already demonstrated performance at the level of human doctoral students in domains such as mathematics \cite{chenDeepMathCreativeBenchmarkEvaluating2025} and programming \cite{jainLiveCodeBenchHolisticContamination2024}.

Despite these advances, significant challenges remain when applying LLM-based systems to real-world tutoring scenarios—most notably the lack of personalized decision-making. This limitation arises because many existing intelligent tutoring systems are primarily built upon AI models coupled with fixed prompt-based query–response mechanisms \cite{goldbachIntelligentTutoringSystems2020,kwonBIPEDPedagogicallyInformed2024}. Relying on a single prompt and the model itself restricts the system’s ability to adapt to learners’ individualized needs during human–AI interaction. Moreover, the one-shot nature of current generation methods hampers their effectiveness in multi-turn interactions and continuous decision-making. As a result, conventional approaches struggle to address the complexity of authentic tutoring contexts.

Taking interactive motion tutoring as an example, human coaches are able to make personalized decisions tailored to individual learners’ conditions during training. In contrast, virtual coaching systems often fail to replicate such capabilities. This is largely due to the fact that human coaching expertise rarely exists in structured, textual form suitable for model training; the available data are both limited in scale and insufficient to cover the diversity of real-world scenarios. Consequently, there remains no widely adopted interactive motion tutoring system that delivers a satisfactory user experience.

To address this gap and tackle the challenge of personalized decision-making in motion tutoring, we propose EDMIT, an end-to-end agentic framework designed to enhance the decision-making capacity of AI systems and closely approximate the decision behaviors of human coaches in interactive exercise guidance. Specifically, through our user-centered investigation, we seek to answer three key questions:

\begin{itemize}
  \item \textbf{RQ1}: How can we design a fully automated AI system that constructs the necessary data inputs at low cost from highly limited and unstructured human coaching experience?
  \item \textbf{RQ2}: How can we design an effective decision-augmentation mechanism that adaptively addresses diverse and personalized tutoring needs in interactive motion scenarios?
  \item \textbf{RQ3}: To what extent can AI-based motion tutoring systems effectively simulate the guidance of human coaches—and is there potential for them to surpass human performance? In other words, what is the upper bound of intelligence for such systems?
  \end{itemize}
  
Based on modeling and data collection of real coaching trajectories, we first constructed seed training and evaluation datasets. To address RQ1, we developed a data augmentation framework that expands the seed data to arbitrary scales, thereby meeting the data requirements for building intelligent AI systems. To address RQ2, inspired by the feedback loop principle in human motor learning, we standardized the modeling of all possible decision nodes in interactive motion tutoring. We further systematized these into a complete decision chain, whose execution is driven by the agent, enabling adaptive tutoring services that accommodate diverse and personalized user needs.

To evaluate whether the EDMIT framework can genuinely enhance users’ motor performance, we introduce—for the first time in AI-assisted motion tutoring—a double-blind experimental design. We developed a prototype system implementing EDMIT to serve real users, who were unaware of whether the guidance they received came from a human coach or the AI system. After each tutoring session, participants rated the different conditions. We conducted a controlled user study involving 10 real users and 10 professional coaches, comparing EDMIT against both human coaching and traditional baseline methods (RQ3). Results demonstrate that EDMIT delivers decision-making performance on par with human coaches, achieves state-of-the-art results relative to existing approaches, and significantly improves users’ acceptance and satisfaction with AI-assisted motion tutoring.

In summary, this work explores an end-to-end agentic framework for enhancing decision-making in interactive motion tutoring, bridging the gap between human coaching expertise and AI intelligence to improve users’ learning outcomes. Our contributions are fourfold:
	1.	A methodological framework that introduces a systematic approach to enhancing decision-making in interactive motion tutoring.
	2.	An open-source system that provides both code implementations and a working prototype to facilitate future research and development.
	3.	A controlled user study that evaluates the effectiveness of our AI system design in real-world tutoring contexts.
	4.	Design insights for future AI-driven interactive motion tutoring systems, highlighting pathways toward effective personalization and human–AI interaction.


\bibliographystyle{ACM-Reference-Format}
\bibliography{sample-base}

\appendix

\section{Research Methods}


\end{document}
\endinput